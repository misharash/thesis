\chapter{Conclusions and outlook}
\label{ch:conclusion}

Galaxy redshift surveys will remain a powerful cosmological probe in the near future.
Among the key analyses of DESI DR2 \citep{DESI.DR2.DR2}, only the BAO \citep{DESI.DR2.BAO.lya,DESI.DR2.BAO.cosmo} have been finished, many others are ongoing.
The products of the originally planned 5 years of observations could become available to the DESI collaboration next year.
The plans beyond that include the upgraded DESI-2 and the next-generation Stage V instrument \citep{spectroscopic-roadmap-cosmic-frontier}.
Besides DESI, the Euclid space telescope \citep{Euclid-overview} is already collecting data, and we hope to see quality spectroscopic products soon.
The Roman Space Telescope \citep[also known as the Wide-Field Infrared Survey Telescope and the Joint Dark Energy Mission][]{WFIRST-2019} should be launched into space soon.

A major part of this thesis (\cref{ch:RascalC}) was dedicated to the development of fast and reliable semi-analytical covariance matrices for 2-point correlation functions of cosmological point tracers.
This approach is much cheaper (both in computing and human time resources) and more flexible than relying on sample variations in large suites of approximate simulations.
Accordingly, it enabled or enhanced many tests of the galaxy and quasar BAO methodology for DESI DR1 \citep{DESI2024.III.KP4}, and then considerably streamlined and sped up the DR2 BAO analysis \citep{Y3.clust-s1.Andrade.2025,DESI.DR2.BAO.cosmo}.
We are looking to further applications of the particular and more generic experience, besides the reiteration for the upcoming DESI Y3 BAO.

Our first natural interest is the adjacent full-shape galaxy clustering analysis \citep[e.g,][]{DESI2024.V.KP5}.
It considers more aspects than only the scale of the BAO feature, including redshift-space distortions and signals from the matter-radiation equality scale.
Accordingly, this analysis informs us much better about the growth of cosmic structure, regulated by the nature of dark matter and gravity.

The advantages come with challenges, such as higher-dimensional data vectors and accordingly covariance matrices (harder to estimate), and the need for more precise non-linear models with more parameters and more degeneracies between them.
The future analysis of DR2 will likely include correlation functions or higher-point statistics to help break such degeneracies.
It may be feasible to extend the covariance framework for these cases \citep[similarly to][]{rascalC-legendre-3}.
We have already suggested a way to correct the simulation-based covariance for power spectra for the 2024 full-shape analysis \citep{DESI2024.V.KP5,DESI2024.II.KP3} using the \rascalc{} semi-analytical covariance matrices.

The official DESI DR1 full-shape analysis \citep{DESI2024.V.KP5} has been conducted in Fourier space (power spectra instead of correlation functions).
We understand it was motivated by concerns that the precision of the modeling, naturally conducted in Fourier space \citep{KP5s1-Maus}, may be partially lost in the Fourier transformation.
However, \cite{KP5s5-Ramirez} reported a high quality of configuration-space (correlation function) modeling and analysis.

\cite{DESI2024.V.KP5} also combined the DR1 BAO post-reconstruction results \citep{DESI2024.III.KP4} with full-shape power spectra, estimating their covariance via mocks.
Analytical or semi-analytical covariance between the reconstructed correlation function and pre-reconstruction power spectrum would be beneficial, but it is very challenging.
It might be feasible to compute the covariance between pre- and post-reconstruction correlation functions similarly to \cref{subsec:post-recon}.
Additionally, we may improve the early implementation of the \rascalc{} covariance between modified\footnote{The difference from the conventional power spectra should be small at small scales.} power spectra and correlation functions \citep{HIPSTER-cov}.

Our results also include a generic covariance matrix validation strategy, not specific to the numeric methods.
In \cref{subsec:comparison-measures,sec:param-space-fisher} we have developed a brief, quantitative, and statistically interpretable framework for comparison with simulation-based covariances.
These methods can be applied to a wide variety of observables besides correlation functions, from galaxy power spectra to CMB instrumental noise.

Higher-point statistics could also be used in future DESI measurements of primordial non-Gaussianity \citep[building upon the completed analysis of DR1 by][]{ChaussidonY1fnl}.
The dimensionality of the data vector grows fast with the number of points, and simulation-based covariances then require proportionally more mock realizations.
The semi-analytical covariances can help to maintain high accuracy at low computation cost.

Another promising direction is the joint analysis of multiple distinct galaxy types --- different tracers of the large-scale structure.
In DESI, several target types overlap in volume, most notably Luminous Red Galaxies \citep[LRG,][]{LRG.TS.Zhou.2023} with Emission Line Galaxies \citep[ELG,][]{ELG.TS.Raichoor.2023} in the redshift range $0.8<z<1.1$, and ELG with quasars \citep{QSO.TS.Chaussidon.2023} in $1.1<z<1.6$.
Additionally, the Bright Galaxy Survey \citep{BGS.TS.Hahn.2023} includes a wide variety of objects and can be separated into multiple sub-types, e.g., red and blue.
Cross-correlations of different galaxies can break parameter degeneracies and effectively increase the measurement precision.
They can also serve as a consistency check to ensure the different samples reflect the same underlying structure \citep[e.g., following][]{LSS-color-dependent-stochasticity}, with the potential to uncover new cosmological tensions.

If the overlapping tracers are analyzed completely separately, valuable information from their cross-correlation remains unused.
Moreover, the two samples are not independent but correlated to some degree, which needs to be quantified.
The full analysis requires high-dimensional covariance matrices between all the auto-correlation and cross-correlation functions, and our semi-analytical framework provides them (in \cref{sec:cov-estimation-extra}).
We have contributed a study of the optimal combination of DESI DR1 LRG and ELG into a single combined tracer \citep{KP4s5-Valcin}, which has been used for the 2024 BAO measurements \citep{DESI2024.III.KP4}.

Our work on the DESI DR1 and DR2 BAO measurements empowered very intriguing cosmological findings \citep{DESI2024.VI.KP7A,DESI.DR2.BAO.cosmo}.
We summarized some of them in \cref{ch:DESI-key}, focusing on dynamic dark energy and Hubble tension (which is not resolved by the dynamic dark energy preferred by DESI).
This motivates us to become more involved in building and testing cosmological models capable of explaining either or both.

We summarized our earlier exploration of a promising Hubble tension relief in \cref{ch:H0-clumping}.
It was a follow-up on the idea of small-scale baryon clumping at recombination \citep{JP20}, possibly induced by primordial magnetic fields \citep{PMF11,PMF13,PMF19}, which gives a higher value of the Hubble constant from CMB even without new physics, and reduces the $S_8$ tension.
Whereas alternative recombination does not require fundamentally new physics, quantifying the uncertainty in known effects is also crucially important.
We conducted a detailed follow-up study with a more flexible and general model for inhomogeneous recombination.
We determined that with data available at the time, the preference for the effect depends significantly on the assumed a priori distributions for clumping parameters, the amount of external Hubble constant measurements, and the inclusion of additional probes like BAO \citep[a conclusion similar to][]{sound-horizon-not-enough}.
We found that the strongest constraints on alternative recombination come from high multipoles of the CMB power spectrum, the ``tail'' affected by Silk damping.
We showed that the future CMB experiments, Simons Observatory \citep{SO} and CMB-S4 \citep{CMBS4,CMBS4white}, will give decisive evidence for or against clumpy recombination.
It may be interesting to revisit the idea with the newest measurements, although some other groups have been working on alternative recombination more actively \citep[e.g.,][]{recombination-reconstruction-Lynch}.

Synergies between advanced CMB and galaxy redshift samples are very promising.
For example, joint analysis of CMB lensing with DESI galaxies has given tighter constraints on the growth of cosmic structure \citep{ACT-lensingxDESI-LRG-structure-formation-Kim,ACT-lensingxDESI-LRG-structure-formation-Sailer} and primordial non-Gaussianity \citep{DESI-QSOxPlanck-lensing-PNG,DESI-LRGxPlanck-lensing-PNG}.
As well as the number of spectra with DESI, the measurements of secondary CMB anisotropies are improving rapidly with the current and next CMB experiments \citep{ACT-lensing-DR6,ACT-maps-DR6,SO,CMBS4,CMBS4white}.
Joint analysis of different measurements has high potential to improve precision, provide valuable information on standing tensions, and discover new discrepancies.

In \cref{ch:tSZ-selection-LRG} we are investigating new techniques using thermal Sunyaev-Zeldovich (tSZ) data instead.
This work is partially inspired by known ideas for leveraging information from different environments: density-split \citep[e.g.][]{density-split-clustering-constrain-nuLCDM} and density-marked \citep[e.g.][]{density-marked-CF-MG} clustering, which deliver tighter cosmological constraints than the conventional 2-point correlation functions or power spectra.
Strong tSZ signals are associated with distinct environments --- large galaxy clusters.
We are grouping DESI luminous red galaxies \citep{LRG.TS.Zhou.2023} by the signal-to-noise ratio in the tSZ $y$ parameter map.
We find that with increasing tSZ signal-to-noise, there are significant increases in large-scale galaxy bias (allowing multi-tracer benefits discussed above), the velocity dispersion \citep[allowing Finger-of-God mitigation for more precise and robust modeling, e.g.][]{removing-FoG} and the number of neighbors \citep[which could help inform the galaxy multiplet studies, e.g.][]{DESI-galaxy-multiplets-tidal-field}.

Thus, we demonstrate that there is valuable cosmological information in the tSZ map beyond the individual clusters.
We hope our preliminary findings are sufficient to draw attention to the topic.
We are going to continue developing the adjacent analysis techniques and encourage others to join us.