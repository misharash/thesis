First, I thank Daniel for being the best advisor I could wish for throughout my time at Harvard.
You have provided invaluable guidance and support while I was navigating grad school, Harvard, the Center for Astrophysics, DESI, and the general scientific community.
You have also devoted a lot to making the Astronomy department a better place as chair and a faculty member, including the amazing introduction to observational astronomy during a yearly trip to Arizona telescopes.
I have learned so much from you, and I still have a long way to go.

I deeply appreciate Cora Dvorkin co-advising my first cosmology project (on the clumpy recombination), teaching the cosmology course, and staying connected beyond the advisory committee.
Julian Mu\~noz was a precious hands-on mentor during this project, helping me a lot to navigate my first cosmology research during the remote COVID year.

John Kovac and Doug Finkbeiner have also been on my advisory committee from the beginning, and I learned much from their different perspectives.
John stressed the importance of a good explanation, the drawbacks of assuming too much prior knowledge, and the value of practicing talks to broader audiences.
Doug often shared my joy of tackling a challenging problem for its own sake, and he also helped organize a wonderful day trip for me and other students on the Hawai'i island after the DESI meeting.

I thank Ravi Sheth for promptly accepting the invitation to be the external examiner and for contributing many interesting questions and suggestions at the private defense.
I am looking forward to our future meetings.

Oliver Philcox also played a significant role in my PhD journey by putting together many cool codes, including \rascalc{}, and two iterations of EFTofLSS lectures.

I deeply value the shared knowledge, experience, interest and care for graduate students from the director of the Center for Astrophysics, Lisa Kewley, and the active faculty of the Astronomy department, particularly Charlie Conroy, Ramesh Narayan, Xingang Chen, Karin \"Oberg, Ashley Villar, Liam Connor and Lars Hernquist.

I have been honored and delighted to work in a very creative and supportive environment of the Eisenstein research group.
I thank my inspiring senior peers: Claire Lamman, Tanveer Karim, Boryana Hadzhiyska, Nina Maksimova, and Lehman Garrison\footnote{We did not overlap at Harvard, but connected through Slack and the Michigan Cosmology School.};
supportive junior peers: Hanyue Wang, Alex Johnson, Zihao Wu, Catherine Miller, Kyle Boone;
and great postdocs around the group: Georgios Valogiannis and Carol Cuesta-Lazaro.
I particularly enjoyed the many shared experiences with Claire, her passion for and great abilities in clear science communication have been a great motivation, and we shared many great fun moments around Harvard and at DESI meetings.
I am looking forward to our continued collaboration and friendship at the Ohio State University.

I have had great opportunities and support through the Dark Energy Spectroscopic Instrument (DESI) collaboration.
I am thankful to the DESI leadership, my mentors and peers, particularly Hee-Jong Seo, Ashley Ross, Arnaud de Mattia, Nikhil Padmanabhan, Florian Beutler, H\'ector Gil-Mar\'in, Pauline Zarrouk, Sesh Nadathur, Dragan Huterer, Will Percival;
Ot\'avio Alves, Uendert Andrade, Ant\'on Baleato Lizancos, Xinyi Chen, Daniel Forero-S\'anchez, Cristhian Garcia-Quintero, Sven Heydenreich, Jiamin Hou, Alex Krolewski, Juan Mena-Fern\'andez, Jeongin Moon, Enrique Paillas, Alejandro P\'erez-Fer\'nandez, Bernie Ried Guachalla, Christoph Saulder, David Valcin.

The great Astronomy graduate student community has been a source of immense support and pleasure.
I particularly appreciate
Hyerin Cho, Shelley Cheng, Alexia Simon, Chris Shallue, MJ Park;
Lieke van Son, Gus Beane, Mike Foley, Charles Law, Miranda Eiben, Floor Broekgaarden, Andrew Saydjari, Jesse Han, Justina Yang, Nayantara Mudur, Zack Murray, Daniel Palumbo, Ioana Zelko;
Kevin Ortiz Ceballos, Karthik Yadavalli, Maryam Hussaini, Vedant Chandra, Rebecca Woody,
Olga Borodina, Ana Sofia Uzsoy, Mouza Almualla, Peixin Zhu, Erandi Chavez,
Ralf Konietzka, Matthew Leung, Abby White,
Tintin Nguyen and Qijia Zhou.
I also value connections with Physics students who worked with Cora: Prish Chakraborty, Gemma Zhang, Anmol Raina, and Chandrika Chandrashekar.

I am deeply grateful to Peg Herlihy, our department administrator, for her constant support, care, invaluable knowledge and experience.
I also thank our graduate coordinators: Robb Scholten (past) and Mark Palmer (current), and CfA administrators Lisa Catella and T.J. Martin.
I am grateful for the international student support provided by the Harvard International Office.

The Derek Bok Center for Teaching and Learning, and particularly Sarah Emory, helped me a lot with English speaking and teaching culture.
The Harvard Horizons symposia have also been a major inspiration.

I am deeply thankful to Vasily Beskin for organizing a range of unofficial astrophysics courses and astro coffee (all aimed at first-year undergraduates), giving me first research opportunities on radio pulsars in my second year, continuing collaboration while I was in Tel Aviv, and supporting my graduate school applications.
I thank Sergey Pilipenko for the unofficial cosmology course introducing correlation function analyses during my first year at MIPT.
I am grateful to Sasha Philippov for co-advising my first radio pulsar project, helping me settle professionally in Tel Aviv and then supporting my graduate school applications.
I greatly appreciate my groupmates and research peers, Alisa Galishnikova and Egor Novoselov, who gave valuable information and advice about US graduate school applications.
During a summer practice at the Special Astrophysical Observatory (in the North Caucasus), I gained a lot of inspiration and encouragement, for which I particularly thank Grigory Beskin and Oleg Verkhodanov.
I owe much of my basis in classical and quantum field theory and general relativity to Emil Akhmedov.

My resolve to pursue cosmology research was solidified at the 15th School of Modern Astrophysics (2019).
I thank the organizers and the lecturers, including Blake Sherwin, Guilherme Pimentel, Andrei Kravtsov, and Mikhail Medvedev.

I am grateful to Omer Bromberg for being an attentive advisor at Tel Aviv University and likewise supporting my graduate school applications.
I value the support and company provided by Lev Pustil'nik and Anna Chashkina.
My favorite advanced elective courses were paradoxes in quantum (paradoxes in physics more generally and quantum theory foundations) by Lev Vaidman and continuum theory (hydrodynamics) by Yacov Kantor.

I fondly remember the beginning of my scientific journey during the 5 years at Richelieu Lyceum, a high school in Odesa specializing in math and natural sciences.
Pavlo Viktor, our physics and astronomy teacher, fueled my love for the subjects, set a great communication example during his lessons, and greatly contributed to the extracurricular activities like the Ecological Expeditions.
Vadim Manakin, my advanced physics teacher and olympiad coach, has always emphasized thinking outside of the box and good humor.
Olga Korenovska, our homeroom and math teacher, as well as deputy principal for scientific activities, has always guided us with high professionalism and deep care.
I am thankful to the former director of the Lyceum, Olga Palladiy, and the current director, Valerii Koleboshin, who also introduced me to the Young Physicists Tournament as early as the 7th grade.
I thank our Young Physicists Tournament team mentors, Anastasia Maslechko, Tetyana Britavska, and Ivan Krychun, for their devotion and setting an example of scientific work and collaboration.

I thank Borys Kremiskyi and Ihor Anisimov for organizing the Ukrainian International Physics Olympiad (IPhO) team, and Stanislav Vilchynskyi for his fascinating lectures and problems on relativity and basic cosmology during our training and selection camps.

I owe my life to the people who organized and carried out my cancer treatment: Roman Goldman, Dr. Odelia Gur, and Dr. Freddy Aviv of Tel Aviv Sourasky Medical Center.
I am deeply thankful to Dr. Natalie Buchkinsky, my primary care provider in Israel, for her sincere support and care ever since.
I am grateful to our family friends who helped me and my parents settle in Israel during that frightful time:
Jenny Batya Spektor and Yosef Spektor,
Leonid and Irina Krasnopolsky,
Sergey, Zhanna, Dmitry and Ora Levin.
I appreciate the support from the Jewish community in Odesa in arranging our emergency repatriation.

I am grateful for the occasional high-stakes strategic battles in Heroes of Might and Magic III (Horn of the Abyss) with my school friends, Daniil (Danya) Selikhanovych and Alexandr (Sasha) Marunchenko; I shall not forget that ``lots'' is 40 (scorpicores).
I also greatly enjoyed video chats with our MIPT friends, Evgeny (Zhenya) Karkaryan and Ilya Simakov, which Sasha suggested during COVID.
I am still eagerly waiting for an opportunity to meet all together in person.
I also appreciated diverse discussions with my other school and college friend, Mykola (Kolya) Kishmar, and was happy to walk around Boston and Cambridge during his visit with a friend.

I am thankful to my Ukrainian IPhO peer and friend, Oleksandr (Sasha) Shumaiev, for connecting me with the young Ukrainian community and his other friends at MIT and Harvard.
The casual meetings with a variety of board games (including Bang, bringing back our IPhO prep days; Secret Hitler, and Codenames) and a spontaneous road trip to Plymouth (MA) and Providence (RI) have been delightful.

I thank my devoted parents, without whom none of this would be possible.
They worked hard to first teach me directly and then emphasized the importance of high-quality education.
They always placed me first and foremost, and undertook an emigration for my sake, leaving behind their long life in Odesa.
Even after they stopped understanding most of my studies, they have always been ready to listen, share my joys and troubles, give valuable advice, help, support, and encouragement.